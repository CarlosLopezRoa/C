\documentclass[a4paper,10pt]{article}
%\documentclass[a4paper,10pt]{scrartcl}
\usepackage{cite}

\usepackage[spanish]{babel}
\decimalpoint 
\usepackage{amsmath}
\usepackage{amssymb}
\usepackage{graphicx}
%\usepackage{/home/vargas/styles/mathshorthand}

%\pagestyle{empty}
%\setlength{\parindent}{0in}
%\setlength{\parskip}{0.1in}
%\setlength{\unitlength}{0.1cm}
\setlength{\oddsidemargin}{0in}

%\setlength{\evensidemargin}{0.5in}
\setlength{\topmargin}{-.8in}
%\setlength{\headsep}{0.1in}}}
%\setlength{\headheight}{0in}
%\setlength{\footskip}{0.5in}
\setlength{\textheight}{10in} \setlength{\textwidth}{6.5in}
%\setlength{\columnsep}{0.2in}
\usepackage{color,hyperref}
\definecolor{darkblue}{rgb}{0.0,0.0,0.3}
\hypersetup{colorlinks,breaklinks,
            linkcolor=darkblue,urlcolor=darkblue,
            anchorcolor=darkblue,citecolor=darkblue}
% For \url{SOME_URL}, links SOME_URL to the url SOME_URL
\providecommand*\url[1]{\href{#1}{#1}}
% Same as above, but pretty-prints SOME_URL in teletype fixed-width font
\renewcommand*\url[1]{\href{#1}{\texttt{#1}}}


\newcommand{\bm}[1]{\boldsymbol{#1}}
\newcommand{\bh}[1]{\boldsymbol{\hat{#1}}}
\newcommand{\bt}[1]{\boldsymbol{\tilde{#1}}}
\newcommand{\bbar}[1]{\boldsymbol{\bar{#1}}}
\newcommand{\mbf}[1]{\ensuremath{\mathbf{#1}}}
\newcommand{\ode}[2]{\ensuremath{\frac{\mathrm{d} #1}{\mathrm{d} #2}}}
\newcommand{\odet}[2]{\ensuremath{\tfrac{\mathrm{d} #1}{\mathrm{d} #2}}}
\newcommand{\oden}[3]{\ensuremath{\frac{\mathrm{d}^#3 #1}{\mathrm{d} #2^#3}}}
\newcommand{\pde}[2]{\ensuremath{\frac{\partial #1}{\partial #2}}}
\newcommand{\pdet}[2]{\ensuremath{\tfrac{\partial #1}{\partial #2}}}
\newcommand{\pden}[3]{\ensuremath{\frac{\partial^{#3} #1}{\partial
      #2^{#3}}}}
\newcommand{\sub}[1]{\ensuremath{_{\rm{#1}}}}
\newcommand{\arriba}[1]{\ensuremath{^{\rm{#1}}}}
%
\newcommand{\N}{\ensuremath{\mathbb{N}}}
\newcommand{\R}{\ensuremath{\mathbb{R}}}
\newcommand{\C}{\ensuremath{\mathbb{C}}}
\newcommand{\ee}[1]{\ensuremath{\mathrm{e}^{#1}}}
\newcommand{\hdos}{\ensuremath{\mathrm{H}_2}}
\newcommand{\COdos}{\ensuremath{\mathrm{CO}_2}}
\newcommand{\ATP}{\ensuremath{\mathrm{ATP}}}

\newcommand{\dt}{\ensuremath{\mathrm{d}t}}
\newcommand{\dtau}{\ensuremath{\mathrm{d}\tau}}
\newcommand{\DV}{\ensuremath{\Delta V}}
\newcommand{\Li}{\mathcal {L}^{-1}}
\newcommand{\Lin}[1]{\mathcal {L}^{-1}\left[{#1}\right]}

\DeclareMathOperator{\sign}{\mathrm{sign}}

\newtheorem{remark}{Remark}
\usepackage[utf8x]{inputenc}
%%

\title{Examen 1\\ Computación 3}
\author{Carlos López Roa}
\date{\today}
\pdfinfo{%
  /Title    ()
  /Author   (CLR)
  /Creator  ()
  /Producer ()
  /Subject  ()
  /Keywords ()
}
\begin{document}
\maketitle
%%% Content
\begin{enumerate}
\item \begin{itemize}
\item ++n;
\item n++;
\item n+=1;
\item n=n+1;
\end{itemize}
\item
 \begin{itemize}
\item long int a=7865;
\item long a=7865;
\end{itemize}
\item \begin{itemize}
\item F
\item V
\item V
\end{itemize}
\item 
\begin{itemize}
\item Asumiendo $\backslash \&$ como $\&$\\
00000001\\
(1)
\item 01110111\\
(119)
\item 01110110\\
(118)
\item 10101000\\
(168)
\end{itemize}
\item else no lleva ;
\item \begin{verbatim}
#include <stdio.h>\\
#include <math.h>
#define max 126
main(){
    int i;
    for (i=32; i<max; ++i) {
        if ((i%16)==0) {
            putchar(i); 
            printf("\n");}
        else 
            putchar(i);
    }
    printf("\n");
}
 \end{verbatim}
\item 
Este programa ingresa un \emph{unsigned integer}, a  dicho integer se le aplica la máscara $1<<15$ que es 1000000000000000 de forma iterativa, en cada iteración al integer se le aplica $integer<<=1$ que es $integer = integer << 1$ o mejor dicho se recorre a la izquierda en una posición. Se realizan 16 iteraciones. Si sucede que al aplicar la máscara $1<<15 $ el resultado es cero entonces se le suma uno a una variable llamada total. Al final la función regresa 1 si el total es divisible entero entre 2 y 0 en otra circunstancia. \\
En resumen la función aplica una máscara 16 veces, en cada ocasión recorre la entrada a la izquierda en una posición. Si el resultado de la máscara es igual a cero, incrementa en uno una variable. Si esa variable es divisible entre 2 regresa 1,  regresa 0 \emph{else}. 
\item \begin{itemize}
\item Como el argumento de putchar es $'1'$ y $'0'$ la salida es $1$ y $0$. El uso de putchar es análogo al de printf, sólo sirve para imprimir los valores que se piden. 
\item \begin{verbatim}
/* Printing an unsigned integer in bits */
#include <stdio.h>
#include <math.h>

main() {
    unsigned x;
    void displayBits(unsigned);
    printf("Enter an unsigned integer\n");
    scanf("%u",&x);
    displayBits(x);
    return 0;
}
void displayBits(unsigned value)
{
    unsigned c, displayMask = 1 << 31;
    printf("%10u = ",value);
    for(c=1; c <= 32; c++)
    {
        putchar(value & displayMask ? '1' : '0');
        value <<=1;
        if(c%16 == 0)
            putchar(' ');
}
        putchar('\n');
    }

\end{verbatim}
\end{itemize}
\item \begin{verbatim}
#include <stdio.h>
#include <math.h>
int power2a(int,int);
void displayBits(unsigned);
main() {
    unsigned x,b;
    printf("Enter an unsigned integer\n");
    scanf("%u",&x);
    printf("Enter a integer power to raise \n");
    scanf("%u",&b);
    printf("Resultado %d\n", power2a(x,b));
    printf("En binario \n");
    displayBits(power2a(x,b));
    return 0;
}
int power2a(int number,int pow){
    int r;
    r=number<<(pow-1);
    return r;
}
void displayBits(unsigned value)
{
    unsigned c, displayMask = 1 << 15;
    printf("%10u = ",value);
    for (c=1; c <= 16; c++)
    {
        putchar(value & displayMask ? '1' : '0');
        value <<=1;
        if (c%8 == 0)
            putchar(' ');
    }
    putchar('\n');
}
\end{verbatim}
Este programa usa $(pow-1)$ y no $pow$ para que $n^1=n$ y no a $2n$
\end{enumerate}




\end{document}